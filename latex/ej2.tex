\section{Ejercicio 2}
    % 1. Describir detalladamente el problema a resolver dando ejemplos del mismo y sus soluciones.
    \subsection{Descripción del problema}
        $N$ soldados de Freezer estan parados en distintos puntos $(X_i,Y_i)$ sobre nuestro planeta y estan dispuestos a acabar con toda la humanidad. Para esto, Goku desea lanzarles algunas Genkidamas que puedan acabar con ellos. Los puntos cumplen con la propiedad $ X_1 > X_2 >
        . . . > X N ≥ 0 y 0 ≤ Y_1 < Y_2 < . . . < Y N$ . Goku quiere lanzar las Genkidamas a puntos donde hay enemigos, por lo que si en un punto no hay un enemigo no puede lanzar una Genkidama a ese punto (si puede lanzarla si habia un enemigo que ya fue destruido por Goku con una Genkidama previa). Una Genkidama lanzada al punto $(X,Y)$ destruye a todos los enemigos que estan en el rectangulo con lados paralelos a los ejes y extremos en $(0, 0) y (X + T, Y + T )$
    % 2. Explicar de forma clara, sencilla, estructurada y concisa, las ideas desarrolladas para la resolución del problema. Utilizar pseudocódigo y lenguaje coloquial (no código fuente). Justificar por qué el procedimiento resuelve efectivamente el problema.
    \subsection{Solución propuesta}
       
    % 3. Deducir una cota de complejidad temporal del algoritmo propuesto y justificar por qué el algoritmo cumple la cota dada. Utilizar el modelo uniforme.
    \subsection{Complejidad teórica}
         
       

    % 4. Dar un código fuente claro que implemente la solución propuesta. Se deben incluir las partes relevantes del código como apéndice del informe impreso entregado.

    % 5. Realizar una experimentación computacional para medir la performance del programa implementado. Usar un conjunto de casos de test en función de los parámetros de entrada, con instancias aleatorias e instancias particulares (de peor/mejor caso en tiempo de ejecución, por ejemplo). Presentar en forma gráfica una comparación entre los tiempos medidos y la complejidad teórica calculada y extraer conclusiones.
    \subsection{Experimentación}
