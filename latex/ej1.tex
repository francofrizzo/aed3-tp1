\section{Ejercicio 1}
    % 1. Describir detalladamente el problema a resolver dando ejemplos del mismo y sus soluciones.
    \subsection{Descripción del problema}
        Gokú debe entrenar un ejército de $N$ guerreros para enfrentarse a Majin Boo. Para lograr ese objetivo, desea organizar peleas entre sus luchadores. En cada pelea, los guerreros serán divididos en dos bandos que pelearán entre sí. El objetivo es que, para cualquier par de luchadores, haya al menos una pelea en la que se encuentren en bandos distintos. Al mismo tiempo, el número total de peleas debe ser el mínimo necesario.

        La resolución del problema consiste en elaborar un programa que recibe como entrada el valor de $N$, y devuelve la cantidad mínima de peleas que deben llevarse a cabo para cumplir el cometido de Gokú y a continuación, por cada una de las peleas, una línea indicando el bando al que pertenece cada uno de los luchadores durante la misma. De haber más de una solución posible, puede devolverse cualquiera de ellas.

        Por ejemplo, si el programa recibe como entrada el valor \texttt{8}, cualquiera de las siguientes tres salidas sería correcta:

        \begin{center}\begin{tabular}{l @{\hskip 2em} | @{\hskip 2em} l @{\hskip 2em} | @{\hskip 2em} l}
            \texttt{3}               & \texttt{3}  &              \texttt{3}               \\
            \texttt{1 1 1 1 2 2 2 2} & \texttt{1 1 2 2 2 1 2 1} & \texttt{2 1 1 1 2 1 2 2} \\
            \texttt{1 1 2 2 1 1 2 2} & \texttt{1 1 2 1 2 2 1 2} & \texttt{1 2 1 1 2 2 2 1} \\
            \texttt{1 2 1 2 1 2 1 2} & \texttt{2 1 2 1 1 1 2 2} & \texttt{1 1 2 1 2 2 1 2} \\
        \end{tabular}\end{center}

    % 2. Explicar de forma clara, sencilla, estructurada y concisa, las ideas desarrolladas para la resolución del problema. Utilizar pseudocódigo y lenguaje coloquial (no código fuente). Justificar por qué el procedimiento resuelve efectivamente el problema.
    \subsection{Solución propuesta}
        La idea central en la resolución del problema fue la siguiente: en cada pelea, Gokú debe tratar de que luchen entre sí la mayor cantidad posible de guerreros que no se hayan enfrentado en una pelea anterior. En particular, en la primera pelea, querrá dividir los bandos de forma tal que luchen entre sí tantos guerreros como sea posible.

        La idea intuitiva es que la solución óptima consiste en minimizar la diferencia entre las cantidades de guerreros de ambos bandos. Por ejemplo, si Gokú pone en un bando a un único guerrero y en el otro a los $N - 1$ guerreros restantes, los pares de guerreros que se enfrentarán en esa pelea serán $N - 1$, ya que todos los guerreros del segundo bando lucharán contra el que quede en el primero, pero nunca pelearán entre sí. Si, en cambio, decide poner $2$ guerreros en un bando, cada uno de ellos se enfrentará con los $N - 2$ del bando opuesto. Así, la cantidad de parejas de guerreros que se enfrenten será $2 (N - 2)$, mejorando el resultado anterior.
        
        En general, el conjunto de pares de guerreros que se enfrentan en cada pelea es el producto cartesiano de ambos bandos. Si el primer bando tiene $k$ guerreros, el otro tendrá $N - k$, por lo que el cardinal de este producto será $N (N - k) = N^2 - Nk$. Es sencillo ver que este producto es máximo cuando $k = \frac{N}{2}$.

        A la hora de resolver el problema, se utilizó la técnica de programación \emph{dividir y conquistar}, volcando de forma casi directa el razonamiento recién expuesto. Si se deja momentáneamente de lado el requerimiento de que la cantidad de peleas efectuadas sea mínima, el problema puede resolverse de esta forma:

        \begin{enumerate}
            \item Se comienza dividiendo a los peleadores en dos subconjuntos $A$ y $B$, que constituirán los bandos que se enfrentarán en la primera pelea.
            \item A continuación, hay que asegurarse de que los guerreros que quedaron en el mismo subconjunto se enfrenten en las peleas posteriores. Esto puede hacerse aplicando el algoritmo de forma recursiva sobre ambos subconjuntos. Así se obtiene, para cada subconjunto, una serie de \emph{subpeleas} que garantiza que todo par de peleadores se enfrenta al menos una vez.
            \item Se concatenan las \emph{subpeleas}, haciendo que la $i$-ésima \emph{subpelea} del subconjunto $A$ ocurra simultáneamente con la $i$-ésima \emph{subpelea} del subconjunto $B$. Si en uno de los subconjuntos deben realizarse más \emph{subpeleas} que en el otro, resulta indistinto en qué bando se encuentran durante las mismas los peleadores del otro subconjunto.
        \end{enumerate}

        Aplicando estos pasos e ignorando el requisito de que la cantidad de peleas sea la mínima posible, Gokú podría obtener, a partir de la entrada $N = 10$, el resultado que se muestra en la Figura \ref{fig:ej1-non-optimal-example}.

        \begin{figure}[h]
          \centering

          \includegraphics{imagenes/ej1-non-optimal-example.pdf} \vspace{1em} \\

          \caption{Ejemplo de solución (no óptima) para el caso $N = 10$. Son necesarias 5 peleas.}
          \label{fig:ej1-non-optimal-example}
        \end{figure}

        Desde luego, nada nos garantiza que, en cada paso de la recursión, Gokú haya tomado las decisiones óptimas. ¿Podría lograrse que todos los peleadores se enfrenten en una menor cantidad de peleas? Aplicando las ideas antes expuestas, y haciendo que en cada paso de la recusión los subconjuntos $A$ y $B$ tengan $\frac{N}{2}$ guerreros cada uno, se obtiene una mejor solución, como muestra la Figura \ref{fig:ej1-optimal-example}.

        \begin{figure}[h]
          \centering

          \includegraphics{imagenes/ej1-optimal-example.pdf} \vspace{1em} \\

          \caption{Ejemplo de solución óptima para el caso $N = 10$. Solo hacen falta 4 peleas.}
          \label{fig:ej1-optimal-example}
        \end{figure}

        

    % 3. Deducir una cota de complejidad temporal del algoritmo propuesto y justificar por qué el algoritmo cumple la cota dada. Utilizar el modelo uniforme.
    \subsection{Complejidad teórica}
        El problema fue tratado como un DyC (divide and conquer), el algoritmo consta en de dividir en dos al N (cantidad de guerreros de la instancia a resolver) y llamar recursivamente las dos mitades. Los casos base son N = 1 y N = 2. Cuando N igual a 1 , se devuelve un array vacio, ya que teniendo un solo guerrero para que todos peleen no tengo que armar ninguna pelea, N igual a 2 , devuelvo el array [1,2] , ya que para que todos pelen teniendo dos guerreros , solo tengo que hacer que peleen entre ellos.
        El caso N > 2, creo la pelea que es que la mitad  pelee con la otra mitad , luego si N es par llamo recursivamente a una mitad y a la otra, luego concateno las peleas de uno con las del otro . Si es impar llamo recursivamente la mitad y la mitad más uno, luego como en el caso de la mitad más 1 , voy a tener una pelea de más la cual voy a concatenar con la ultima de la otra mitad (podria concatenarlo con cualquiera) y devuelvo las peleas concatenadas como solución . 
        Para justificar la complejidad usaremos el Teorema Maestro. Dado que 2 es la cantidad des subproblemas a resolver y tambien las particiones y n es el costo de cada paso en la recursión , el algoritmo cae en el segundo caso , entonces la complejidad es de n log n.



    % 4. Dar un código fuente claro que implemente la solución propuesta. Se deben incluir las partes relevantes del código como apéndice del informe impreso entregado.

    % 5. Realizar una experimentación computacional para medir la performance del programa implementado. Usar un conjunto de casos de test en función de los parámetros de entrada, con instancias aleatorias e instancias particulares (de peor/mejor caso en tiempo de ejecución, por ejemplo). Presentar en forma gráfica una comparación entre los tiempos medidos y la complejidad teórica calculada y extraer conclusiones.
    \subsection{Experimentación}
